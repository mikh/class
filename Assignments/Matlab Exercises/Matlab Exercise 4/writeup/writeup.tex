\documentclass[paper=a4, fontsize=11pt]{scrartcl} % A4 paper and 11pt font size

\usepackage[T1]{fontenc} % Use 8-bit encoding that has 256 glyphs
\usepackage[english]{babel} % English language/hyphenation
\usepackage{amsmath,amsfonts,amsthm} % Math packages
\usepackage{graphicx}

\usepackage{lipsum} % Used for inserting dummy 'Lorem ipsum' text into the template

\usepackage{sectsty} % Allows customizing section commands
\allsectionsfont{\centering \normalfont\scshape} % Make all sections centered, the default font and small caps

\usepackage{fancyhdr} % Custom headers and footers
\pagestyle{fancyplain} % Makes all pages in the document conform to the custom headers and footers
\fancyhead{} % No page header - if you want one, create it in the same way as the footers below
\fancyfoot[L]{} % Empty left footer
\fancyfoot[C]{} % Empty center footer
\fancyfoot[R]{\thepage} % Page numbering for right footer
\renewcommand{\headrulewidth}{0pt} % Remove header underlines
\renewcommand{\footrulewidth}{0pt} % Remove footer underlines
\setlength{\headheight}{13.6pt} % Customize the height of the header

\numberwithin{equation}{section} % Number equations within sections (i.e. 1.1, 1.2, 2.1, 2.2 instead of 1, 2, 3, 4)
\numberwithin{figure}{section} % Number figures within sections (i.e. 1.1, 1.2, 2.1, 2.2 instead of 1, 2, 3, 4)
\numberwithin{table}{section} % Number tables within sections (i.e. 1.1, 1.2, 2.1, 2.2 instead of 1, 2, 3, 4)

\setlength\parindent{0pt} % Removes all indentation from paragraphs - comment this line for an assignment with lots of text

%----------------------------------------------------------------------------------------
%	TITLE SECTION
%----------------------------------------------------------------------------------------

\newcommand{\horrule}[1]{\rule{\linewidth}{#1}} % Create horizontal rule command with 1 argument of height

\title{	
	\normalfont \normalsize 
	\textsc{EC500 - Introduction to Learning From Data} \\ [25pt] % Your university, school and/or department name(s)
	\horrule{0.5pt} \\[0.4cm] % Thin top horizontal rule
	\huge Matlab 4 \\ % The assignment title
	\horrule{2pt} \\[0.5cm] % Thick bottom horizontal rule
}

\author{Mikhail Andreev} % Your name

\date{\normalsize\today} % Today's date or a custom date

\begin{document}
	
	\maketitle % Print the title
	

	%----------------------------------------------------------------------------------------
	%	PROBLEM 1
	%----------------------------------------------------------------------------------------
	
	\newpage
	\section{k-Means vs. Spectral Clustering}
	The k-means approach to the dataset yield the following three graphs for k=2,3,4 respectively:\\\\\\
	\hspace*{-3cm}\includegraphics[scale=0.6]{kmeans_k2}
	\\\\\\
	\hspace*{-2.5cm}\includegraphics[scale=0.6]{kmeans_k3}
	\\\\\\
	\hspace*{-2.5cm}\includegraphics[scale=0.6]{kmeans_k4}
	\\\\\\
	The cyan points in graphs indicate the centroids for each of the clusters. The overall sums of l2 distances between each point and the centroid for the clusters are:
	\\\\
	cluster 1 (red) - circle: 757.41, spiral: 1098.18\\
	cluster 2 (blue) - circle: 721.62, spiral: 1088.39\\
	cluster 3 (green) - circle: 522.01, spiral: 750.26\\
	cluster 4 (black) - circle: 444.34, spiral: 609.27
	\\\\\\
	The eigenvalues of the Laplacian matrices can be seen here:
	\\\\\\
	\hspace*{-3cm}\includegraphics[scale=0.6]{eigenvalues}
	\\\\\\
	After applying spectral clustering we get the following output using symmetric normalization:
	\\\\
	\hspace*{-3cm}\includegraphics[scale=0.8]{symmetric_spectral_clustering}
	\\\\\\
	In the next figure we can see the 3D interpretation of the eigenvectors, showing that all three clusters are clearly separate in eigenspace:
	\\\\
	\hspace*{-4.3cm}\includegraphics[scale=0.55]{3dplots}
	\\\\\\
	If we convert the data into polar coordinates, we see it is much easier for kmeans to classify the circle dataset, but still difficult to classify the spiral dataset.
	\\\\\\
	\hspace*{-3cm}\includegraphics[scale=0.8]{polar_kmeans}
	\\\\\\
	The cyan points in graphs indicate the centroids for each of the clusters. The overall sums of l2 distances between each point and the centroid for the clusters are:
	\\\\
	cluster 1 (red) - circle: 904.57, spiral: 621.46\\
	cluster 2 (blue) - circle: 742.63, spiral: 706.96\\
	cluster 3 (green) - circle: 634.15, spiral: 539.10\\
	cluster 4 (black) - circle: 236.26, spiral: 453.95
	
	\newpage
	\section{Spectral Clustering on Airbnb Data}

	
\end{document}