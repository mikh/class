\documentclass[paper=a4, fontsize=11pt]{scrartcl} % A4 paper and 11pt font size

\usepackage[T1]{fontenc} % Use 8-bit encoding that has 256 glyphs
\usepackage[english]{babel} % English language/hyphenation
\usepackage{amsmath,amsfonts,amsthm} % Math packages
\usepackage{graphicx}

\usepackage{lipsum} % Used for inserting dummy 'Lorem ipsum' text into the template

\usepackage{sectsty} % Allows customizing section commands
\allsectionsfont{\centering \normalfont\scshape} % Make all sections centered, the default font and small caps

\usepackage{fancyhdr} % Custom headers and footers
\pagestyle{fancyplain} % Makes all pages in the document conform to the custom headers and footers
\fancyhead{} % No page header - if you want one, create it in the same way as the footers below
\fancyfoot[L]{} % Empty left footer
\fancyfoot[C]{} % Empty center footer
\fancyfoot[R]{\thepage} % Page numbering for right footer
\renewcommand{\headrulewidth}{0pt} % Remove header underlines
\renewcommand{\footrulewidth}{0pt} % Remove footer underlines
\setlength{\headheight}{13.6pt} % Customize the height of the header

\numberwithin{equation}{section} % Number equations within sections (i.e. 1.1, 1.2, 2.1, 2.2 instead of 1, 2, 3, 4)
\numberwithin{figure}{section} % Number figures within sections (i.e. 1.1, 1.2, 2.1, 2.2 instead of 1, 2, 3, 4)
\numberwithin{table}{section} % Number tables within sections (i.e. 1.1, 1.2, 2.1, 2.2 instead of 1, 2, 3, 4)

\setlength\parindent{0pt} % Removes all indentation from paragraphs - comment this line for an assignment with lots of text

%----------------------------------------------------------------------------------------
%	TITLE SECTION
%----------------------------------------------------------------------------------------

\newcommand{\horrule}[1]{\rule{\linewidth}{#1}} % Create horizontal rule command with 1 argument of height

\title{	
	\normalfont \normalsize 
	\textsc{EC500 - Introduction to Learning From Data} \\ [25pt] % Your university, school and/or department name(s)
	\horrule{0.5pt} \\[0.4cm] % Thin top horizontal rule
	\huge Matlab 5 \\ % The assignment title
	\horrule{2pt} \\[0.5cm] % Thick bottom horizontal rule
}

\author{Mikhail Andreev} % Your name

\date{\normalsize\today} % Today's date or a custom date

\begin{document}
	
	\maketitle % Print the title
	

	%----------------------------------------------------------------------------------------
	%	PROBLEM 1
	%----------------------------------------------------------------------------------------
	
	\newpage
	\section{k-Means vs. Spectral Clustering}
	\subsection{Part a}
	Using the .mat data provided we can view the helix and the swiss roll:
	\\\\
	\hspace*{-3cm}\includegraphics[]{helix}
	\\\\
	\hspace*{-3cm}\includegraphics[]{swiss_roll}
	\subsection{Part b}
	Performing KPCA with the linear, polynomial, and rbf kernels returns the following graphs:
	\\\\
	\hspace*{-4cm}\includegraphics[scale=0.7]{3kernels}
	\subsection{Part c}
	None of the methods are fully successful at unrolling the shapes. Overall, the helix seems to be more coherent, although we would ideally want the output in the form of a circle. For the swiss roll, the linear kernel completely fails at unrolling the data, superimposing the 3D image directly onto 2D. The polynomial kernel is not able to unroll the data. The rbf kernel is the closest to unrolling it, but it loses the structure of the data when it is mapped onto 2D.
	\subsection{Part d}
	After creating the k-NN graphs we can visualize them here:
	\\\\
	\hspace*{-3cm}\includegraphics[]{helix_knn}
	\\\\
	\hspace*{-3cm}\includegraphics[]{swiss_knn}
	\\\\
	As can be seen, these closely resemble the shapes seen in part a. If we were to decrease the value of k, the wireframes become more sparse, but the overall shape remains.
	\\\\\\
	After performing the ISOMAP KPCA we have the following 2D result:
	\\\\\\
	\hspace*{-3cm}\includegraphics[]{isomap}
	\subsection{Part e}
	From this graph we see that we have a much better KPCA result using ISOMAP than we did with the other kernels. For the helix graph there seems to be some amount of over-fitting involved causing the blip on the right. If we decrease the value of k, the graph becomes much more smooth. For the swiss roll we see that this is a good 2 dimensional representation of the data. We can clearly see the length of the geodesic distance, while maintaining a close Euclidean distance.
	
\end{document}