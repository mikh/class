\documentclass[paper=a4, fontsize=11pt]{scrartcl} % A4 paper and 11pt font size

\usepackage[T1]{fontenc} % Use 8-bit encoding that has 256 glyphs
\usepackage[english]{babel} % English language/hyphenation
\usepackage{amsmath,amsfonts,amsthm} % Math packages
\usepackage{graphicx}

\usepackage{lipsum} % Used for inserting dummy 'Lorem ipsum' text into the template

\usepackage{sectsty} % Allows customizing section commands
\allsectionsfont{\centering \normalfont\scshape} % Make all sections centered, the default font and small caps

\usepackage{fancyhdr} % Custom headers and footers
\pagestyle{fancyplain} % Makes all pages in the document conform to the custom headers and footers
\fancyhead{} % No page header - if you want one, create it in the same way as the footers below
\fancyfoot[L]{} % Empty left footer
\fancyfoot[C]{} % Empty center footer
\fancyfoot[R]{\thepage} % Page numbering for right footer
\renewcommand{\headrulewidth}{0pt} % Remove header underlines
\renewcommand{\footrulewidth}{0pt} % Remove footer underlines
\setlength{\headheight}{13.6pt} % Customize the height of the header

\numberwithin{equation}{section} % Number equations within sections (i.e. 1.1, 1.2, 2.1, 2.2 instead of 1, 2, 3, 4)
\numberwithin{figure}{section} % Number figures within sections (i.e. 1.1, 1.2, 2.1, 2.2 instead of 1, 2, 3, 4)
\numberwithin{table}{section} % Number tables within sections (i.e. 1.1, 1.2, 2.1, 2.2 instead of 1, 2, 3, 4)

\setlength\parindent{0pt} % Removes all indentation from paragraphs - comment this line for an assignment with lots of text

%----------------------------------------------------------------------------------------
%	TITLE SECTION
%----------------------------------------------------------------------------------------

\newcommand{\horrule}[1]{\rule{\linewidth}{#1}} % Create horizontal rule command with 1 argument of height

\title{	
	\normalfont \normalsize 
	\textsc{EC500 - Introduction to Learning From Data} \\ [25pt] % Your university, school and/or department name(s)
	\horrule{0.5pt} \\[0.4cm] % Thin top horizontal rule
	\huge Matlab 5 \\ % The assignment title
	\horrule{2pt} \\[0.5cm] % Thick bottom horizontal rule
}

\author{Mikhail Andreev} % Your name

\date{\normalsize\today} % Today's date or a custom date

\begin{document}
	
	\maketitle % Print the title
	

	%----------------------------------------------------------------------------------------
	%	PROBLEM 1
	%----------------------------------------------------------------------------------------
	
	\newpage
	\section{Introduction}
	The problem we are seeking to solve is that of image classification. Given a set of handwritten digits, the goal is to determine the value of an arbitrary digit. I seek to accomplish this through the use of neural networks, specifically using Multi-layer perceptron networks and convolutional networks. Neural networks are well adapted for this level of problem because they can find underlying patterns within the images, contributing to classification.
	\section{Literature Review}
	There are many ways to approach this problem, 
	
	\section{Problem Statement}
	
	\section{Implementation}
	
	\section{Experimental Results}
	
	\section{Conclusions}
	
	\section{Description of Individual Effort}
	
	\section{References}
	
	\section{Appendix}
	
	
\end{document}